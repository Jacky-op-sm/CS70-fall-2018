\documentclass[12pt]{exam}
\usepackage[left=3cm, right=3cm, top=3cm]{geometry}
\usepackage{amsmath,amsthm,amssymb}
\usepackage{enumerate}
\usepackage{xcolor}
\usepackage{mathtools}

\newcommand{\red}{\color{red}}

\setlength{\parindent}{0pt}
\title{CS70 HW4}
\date{February 28, 2021}

\begin{document}
\maketitle

\tableofcontents
%\listoffigures
%\thispagestyle{empty}
\pagenumbering{arabic}

\section{Modular Arithmetic Solutions}

\begin{parts}

	\part $2x \equiv 5 \pmod{15}$.
	
	Here since gcd(2, 15) = 1, $2^{-1} \pmod{15}$ is unique and equal to 8. \\
	So $x \equiv 5 * 8  \equiv 40 \equiv 10 \pmod{15}$

	\part $2x \equiv 5 \pmod{16}$
	
	Here since gcd(2, 16) = 2, and $2x = 5 + 16*k$, for some k $\in \mathbb{Z}$. The right side of this equation is odd number, but left side is even. Thus there is no such solution for x.
	
	\part $5x \equiv 10 \pmod{25}$
	
	Here $5x = 10 + 25*k$, for some k $\in \mathbb{Z}$. So $x = 2 + 5*k$ and in the range of [0, 24]. For all possible k, x is in the set of \{2, 7, 12, 17, 22\}.
	
\end{parts}

\section{Euclid’s Algorithm}

\begin{parts}

	\part The procedure is as follow: 
	\begin{align*} 
	gcd(527, 323) &= gcd(323, 204) \\
	&= gcd(204, 109) \\
	&= gcd(109, 95) \\
	&= gcd(95, 14) \\
	&= gcd(14, 11) \\
	&= 1
	\end{align*}
	
	{\red
	\begin{align*} 
	gcd(527, 323) &= gcd(323, 204) \\
	&= gcd(204, 119) \\
	&= gcd(119, 85) \\
	&= gcd(85, 34) \\
	&= gcd(34, 17) \\
	&= gcd(17, 0) \\
	&= 17
	\end{align*}
	}


	\part The procedure is as follow: 
	\begin{align*} 
	gcd(27, 5) &= gcd(5, 2)  &  & [2 = 27 + (-5) * 5]\\
	&= 1 &  & [1 = 5 + (-2) * 2]
	\end{align*}
	Use the right side of equation, we obtain
	\begin{align*} 
	1 &= 5 + (-2) * [27 + (-5) * 5] \\
	&= (-2) * 27 + 11 * 5
	\end{align*}
	So the multiplicative inverse of $5 \pmod {27} $ is 11. 
	
	\part The procedure is as follow: 
	\begin{align*} 
	5x + 26 &\equiv 3 \pmod{27} \\
	5x &\equiv 4 \pmod{27}\\ 
	x &\equiv 44 \pmod{27}\\
	x &\equiv 17 \pmod{27}
	\end{align*}
	
	\part Disprove.
	
	$a$ has no multiplicative inverse mod $c \implies gcd(a, c) \neq 1 $. But it may still have solution. Counter example is $5x \equiv 10 \pmod{25}$, in the section 1 part (c).

\end{parts}

\section{Modular Exponentiation}

\begin{parts}

	\part $13^{2018} \equiv 1 \pmod {12}$.
	
	Since $13 \equiv 1 \pmod {12}$ and $1^{2018} \equiv 1 \pmod {12}$ is obvious.

	\part $8^{11111} \equiv 7 \pmod 9$.
	
	(By Fermat's Equation, )$8^{8} \equiv 1 \pmod 9$. And $11111 \equiv 7 \pmod 8$. And $ 8 * 8^{7} \equiv 1 \pmod 9$, since gcd(8, 9) = 1,  $8^{7} \pmod 9$ is equal to the multiplicative inverse of $8 \pmod 9$, which is 7.
	
	{\red $8^{11111} \equiv 8 \pmod 9$
	
	 Since $8^2 \equiv 1 \pmod 9$ and $8^{11111} = 8^{5555 * 2 + 1} = 8 \pmod 9$}
	
	\part $7^{256} \equiv 4 \pmod {11}$.
	
	By Fermat's Equation, $7^{10} \equiv 1 \pmod {11}$. And $256 \equiv 6 \pmod {10}$, which is to compute $7^{6} \pmod {11}$. $7^{6} = 49 * 49 * 49 \equiv 5 * 5 * 5 \equiv 125 \equiv 4 \pmod {11}$.
	
	\part $3^{160} \equiv 16 \pmod {23}$.
	
	By Fermat's Equation, $3^{22} \equiv 1 \pmod {23}$. And $160 \equiv 6 \pmod {22}$, which is to compute $3^{6} \pmod {23}$. $3^{6} = 9 * 81 \equiv 9 * 12 \equiv 108 \equiv 16 \pmod {23}$.
	
\end{parts}

\section{Euler’s Totient Function}

\begin{parts}

	\part Since p is a prime number, [0, 1, 2 \dots p - 1] are all relatively prime to p. \\
	Thus $\phi(p) = p - 1$.

	\part Since p is a prime, then $p^k$ has only factor of p $\implies$ set S $\{p, 2p, \dots, p^k\}$ are relatively not prime to p and $|S| = p^k / p = p^{k-1}$. \\
	Thus $\phi(p^k) = p^k - p^{k-1} = (p-1)*p^{k-1}$.
	
	\part From part (a), $\phi(p) = p - 1$. Thus $a^{\phi(p)} = a^{p - 1}$, which according to Fermat's Little Theorem, $a^{p - 1} \equiv 1 \pmod p$.
	
	\part To prove
	$$\forall i \in \{1,2,\hdots,k\}, \hspace{0.2cm} a^{\phi(b)} \equiv 1 \pmod{p_i}$$
	The key part is to solve $\phi(b)$.\\
	Since $b = p_1^{\alpha_1}\cdot p_2^{\alpha_2} \hdots p_k^{\alpha_k}$, from Euler's totient function, $\phi(b) = \phi(p_1^{\alpha_1})  \cdot \phi(p_2^{\alpha_2}) \hdots \phi(p_k^{\alpha_k}) = (p_1-1)*p^{\alpha_1 - 1} \cdot (p_2-1)*p^{\alpha_2 - 1}\hdots (p_k-1)*p^{\alpha_k - 1} = (p_1-1)\cdot(p_2-1)\hdots(p_k-1)* K $, for some integer K $\in \mathbb{Z}$. Now since $\phi(b)$ has factors of $(p_i-1), \forall i \in \{1,2,\hdots,k\}$, the proof is done.

\end{parts}


\section{FLT Converse}
\begin{parts}

	\part Since $gcd(n, a) \neq 1$, there exists a common factor, namely $k \ge 2$ that $n = c_0 * k$ and $a = c_1 * k$, for some $c_0, c_1, k \in \mathbb{N}$. Assume the equation $a^{n-1} \equiv 1 \pmod{n}$ holds, meaning that $\exists c_2 \in \mathbb{N}, a^{n-1} = 1 + c_2 * n \implies (c_1 * k)^{n-1} - c_0 *c_2* k = 1$. Since the left side has common factor of k, but the right side is 1, which implies $k=1$, that $gcd(n, a) = 1$, contradiction. 
	
	Thus for every $a$ and $n$ such that $\texttt{gcd}(n,a) \neq 1$, we have $a^{n-1} \not\equiv 1 \pmod{n}$.

	\part From $S(n) =  \{i: 1 \leq i \leq n, \texttt{gcd}(n,i) = 1\}$, we can define two set: \\
	$Q(n) =  \{q: q \in S(n), q^{n-1} \equiv  1 \pmod n\}$; \\
	$P(n) =  \{p: p \in S(n), p^{n-1} \not\equiv  1 \pmod n\}$; \\
	$\mid S \mid = \mid Q \mid +  \mid P \mid$; If we can find $q \in P(n)$, then we can define a map T from Q(n) to P(n), T = p * $q_i$, for $q_i \in Q(n)$.\\
	Now we wanna prove it's one-to-one. 
	\begin{proof}
	Proof by contradiction. 
	
	Say $q_1 \neq q_2$, but $p * q_1  \equiv p * q_2 \pmod n$. Then we have $p * q_1 = p * q_2 + c_0 * n$, which is same as $p * q_1 - p * q_2 = p (q_1 - q_2) = c_0 * n$. But the since gcd(p, n) = 1, $q_1 - q_2$ must be multiple of n, which is impossible. 
	\end{proof}
	Now since it's one-to-one, we obtain $\mid Q \mid$ is at least as large $\mid P \mid$. Thus $\mid Q \mid \ge \mid S \mid / 2 $.
	
	\part $a \equiv b \mod m_1 \implies a - b = c_0 * m_1$, for some $c_0 \in \mathbb{Z}$;\\
	$a \equiv b \mod m_2 \implies a - b = c_1 * m_2$, for some $c_1 \in \mathbb{Z}$;\\
	Since $\gcd(m_1, m_2)=1$, a - b is multiple of $m_1$ and $m_2$, it must also be the multiple of $m_1 m_2$. Thus $a - b = c_2 * m_1 m_2$, for some $c_2 \in \mathbb{Z}$.\\
	So $a \equiv b \pmod{m_1 m_2}$.
	
	\part Since $a \in S(n)$ and $n = p_1 p_2 \cdots p_k$, we have the fact that a is prime to $p_1, p_2, \cdots ,p_k$. Then $a^{n-1} = a^{p_1 p_2 \cdots p_k - 1}$.  \\
	$\forall i \in \{1, 2, \dots, k\}$, since $p_i - 1 \mid n - 1$ and also $a$ is prime to $p_i$, $a^{n-1} = a^{(p_i-1)*K} \equiv 1 \pmod {p_i}$. And since $p_i$ are distinct primes, $gcd(p_i, p_j) = 1$ for $i \neq j$.
	
	So $a^{n-1} \equiv 1 \pmod{n}$ for all $a \in S(n)$.
	
	\part From part (b), number 561 has following properties:
	
	\begin{enumerate}[(1)]
	
		\item 561 = 3 * 11 * 17 where 3, 11 and 17 are distinct primes. 
	
		\item 560 = 2 * 280; 560 = 10 * 56; 560 = 16 * 35 $\implies (3-1)\mid 560, (11-1)\mid 560, (17-1)\mid 560$.
	
	\end{enumerate}
	So for all $a$ coprime with 561, $a^{560} \equiv 1 \pmod{561}$.


\end{parts}


\end{document}