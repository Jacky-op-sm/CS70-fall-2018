\documentclass[12pt]{exam}
\usepackage[left=3cm, right=3cm, top=3cm]{geometry}
\usepackage{amssymb}
\usepackage{amsmath}
\usepackage{enumerate}

\begin{document}
{\Large 3 Propositional Practice} \\[.3cm]

\begin{parts}

\part $(\exists x \in \mathbb{R}) \ (x \notin \mathbb{Q})$

True.

Consider $x = \pi$. $\pi \in \mathbb{R}$, and $\pi \notin \mathbb{Q}$, so the problem is true.

\part $(\forall x \in \mathbb{Z}) \ (((x \in \mathbb{N}) \lor (x < 0)) \land (\neg ((x \in \mathbb{N}) \land (x < 0 ))))$ 

True.

Let $x \in \mathbb{N}$, so $x >= 0$ or $x < 0$, but not both.

If $x >= 0$, then $x$ is a natural number; if $x < 0$, then is negative; $x$ can't be both.

Thus, the proposition is true.

\part $(\forall x \in \mathbb{N})\ ((6 \mid x) \Rightarrow ((2 \mid x) \lor (3 \mid x)))$

True.

Let $x \in \mathbb{N}$, $x = 6 * k$, so $x \in \mathbb{N}$

So $x = 2 * (3k)$ where $3k\in \mathbb{N}$, which means that $ 2 \mid x$

So $((2 \mid x) \lor (3 \mid x))$ is true, which means that the proposition is true. 

\part All real numbers are complex numbers.

True.

Let $x \in \mathbb{R}$, so $x = x + 0 * i$, and sin $x, 0 \in \mathbb{R}$, So by the definition of complex numbers, $x$ is a complex number.

\part Integers that are divisible by 2 or 3 are divisible by 6.

False. 

Give a counter example, take integer $n = 2$, $2 \mid n$ is true but $6 \mid n$ is false.

So by the definition of implication, this proposition is false.

\end{parts}
\end{document}