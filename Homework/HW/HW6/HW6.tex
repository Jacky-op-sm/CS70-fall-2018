\documentclass[12pt]{exam}
\usepackage[left=3cm, right=3cm, top=3cm]{geometry}
\usepackage{amsmath,amsthm,amssymb}
\usepackage{enumerate}
\usepackage{xcolor}
\usepackage{mathtools}

\newcommand{\red}{\color{red}}

\setlength{\parindent}{0pt}
\title{CS70 HW 6}
\date{Mar 6, 2021}

\begin{document}
\maketitle
\section{Polynomial Practice}

\begin{parts}
	
	\part Say function $f$ degree is $d_f$, function $g$ is $d_g$.
	\begin{enumerate}[(i)]
	
		\item at least 0, at most $max\{d_f, d_g\}$. Because $0 \leq degree(f+g) \leq max\{d_f, d_g\}$.
	
		\item at least 0, at most $d_f + d_g$. Because $0 \leq degree(f \cdot g) \leq d_f +  d_g$.
		
		\item at least 0, at most $d_f - d_g$. Because $0 \leq degree(f / g) \leq d_f - d_g$.
		
	\end{enumerate}

	\part 
	\begin{enumerate}[(i)]
	
		\item $0 \leq degree(f \cdot g) \leq d_f +  d_g$. Therefore, $d_f +  d_g = 0 \implies d_f = 0 \lor d_g = 0$. Thus either $f = 0$ or $g = 0$.
		
		{\red {\bf Didn't see the GF(p) in the question.\\ } we can construct $f(x) = x^{p-1} - 1$ and $g(x) = x$, in which $f(x) \equiv 0 \pmod p$ except $x = 0$, and $g(x) \equiv 0 \pmod p$ when $x = 0$, thus $f \cdot g \equiv 0 \pmod p$ in all range.}
		
		Or proof by contradiction.
		\begin{proof}
		Say $f \neq 0 \land g \neq 0$, then $f * g \neq 0$, which contradicts to $f*g = 0$.
		\end{proof}
	
		\item By Fermat's little theorem, for any $x \in \{0, 1, \dots, p - 1\}, x^{p-1} \equiv 1 \pmod p if GF(p)$. \\
		Any degree $y \geq p$ can be expressed $x^y \equiv x^a \pmod p$ where $a \in \{0, 1, \dots, p-1\}$. 
		
		\item $d$ degree needs $d + 1$ points to interpolate. \\
		now (0, a) is fixed $\implies$ needs d points. \\
		All points have p choices $\implies p^d$ number of polynomials. 
	
	\end{enumerate}
	
	\part Given three points, we can def $p(x) = ax^2 + bx + c$ and have equations:
	\begin{align*} 
	c &= 1 \\
	4a + 2b + c &= 2 \\
	16a + 4b + c &= 0
	\end{align*}
	Solving there equations, we have $c = 1, a = -3/8, b = 5/4$. Therefore, $g(x) = \frac{-3}{8}x^2 + \frac{5}{4}x + 1$.\\
	And for polynomials that degree is greater than 2, there are 5 choices for degree 3, $5^2 = 25$ choices for degree 4.\\
	Thus in total : 31 polynomials.
	
	{\red in total:25. The degree 4 situation includes degree 3.}
\end{parts}

\section{The CRT and Lagrange Interpolation}
\begin{parts}

	\part Since $a_2 = 0$, so $x$ is multiple of $n_2$, aka $x = kn_2$. But from $x \equiv a_1 \pmod {n_1}$ and $x*x^{-1} \equiv 1 \pmod {n_1}$, we can construct $x_1 = kn_2*(kn_2)^{-1} = n_2*(n_2)^{-1} \pmod {n_1}$. \\
	Similarly, for $a_1 = 0, a_2 = 1$, we can construct $x_2 = n_1*(n_1)^{-1} \pmod {n_2}$.

	\part For any $a$ and $b$, we can use the solution from part (a), to obtain $X = ax_1 + bx_2$. Therefore, there is at least one solution.
	
	For uniqueness, take $x_1 = n_2*(n_2)^{-1} \pmod {n_1}$ for example, since $gcd(n_1, n_2) = 1$, so the $(n_2)^{-1}$ is unique and make $x_1$ is unique $\pmod {n_1}$ also. $x_2$ is unique as the same reason.
	
	\item 
	\begin{proof}
	
	Proof by Induction on the number of the equation k.
	
	\emph{Base Case}: When $k = 1, x \equiv a_1 \pmod {n_1}$, obviously $x$ exists and is unique.
	
	\emph{Induction Hypothese}: Assume when $k = m$, the equation holds and the result $x_m \equiv a_m \pmod {n_m}$.
	
	\emph{Induction Step}: Use the part b, we know there is a unique solution of the equation 
	\begin{align*} 
	x_m &\equiv a_m \pmod {n_m} \\
	x_{m+1} &\equiv a_{m+1} \pmod {n_{m+1}}
	\end{align*}
	
	so proof is done.
	
	\end{proof}
	
	\part For integer $a, b, a = b * Q + R$ and $a \equiv R \pmod q$.\\
	To mimic such relation, $P(x) = q(x) * Q(x) + R(x)$ and $P(x) \equiv R(x) \pmod {q(x)}$.
	
	To compute $p(x)$ mod $(x-1)$, say $p(x) = a_0 + a_1x + \dots + a_kx^k$. For $x^n$, we can express it as $((x - 1) + 1)^n$, which is $\equiv 1 \pmod {x-1}$. So $p(x) \equiv \sum _{i = 0} ^ k a_i \pmod {x-1}$.
	
	\item Given CRT still holds when replacing $x, a_i$ and $n_i$ with polynomials, now we just need to prove that $(x - x_i)$ are coprime when $x_i$ are pairwise distinct. 
	
	\begin{proof}
	Proof by contradiction.\\
	Say $(x-x_i)$ and $(x- x_k)$ are not coprime when $x_i \neq x_k$. By the definition of coprime in polynomial, there is a degree 1 polynomial, i.e $(x-a)$ dividing both. Since $(x-x_i)$ is degree 1, if it's divided by $(x-a)$, $a$ must be $x_i$. It's also same when $(x - x_k)$, so $a$ must be $x_k$. So $x_i = x_k$, which contradicts to $x_i \neq x_k$. So $(x - x_i)$ are coprime.
	\end{proof} 
	
	The relation with Lagrange interpolation: the solution is exactly the way of Lagrange interpolation construction. Since in points interpolation, $x_i$ are pairwise distinct, so the solution is consistent and unique. 

\end{parts}

\section{Old secrets, new secrets}
The answer is simple, joke on his information of $(1, P(1))$, say give his friends $(1, P(1)')$ where $P(1)' \neq P(1)$. Since there is unique polynomial each with $n$ points with $(1, P(1))$ and $(1, P(1)')$, the former one will get original secret $s$, while the latter one will get $s'$ where $s' \neq s$.


\section{Berlekamp-Welch for General Errors}
\begin{parts}

	\part Degree of E(x): 1;\\
	Degree of Q(x): 3;\\
	$E(x) = x - b. Q(x) = a_0 + a_1x + a_2 x^2 + a_3 x^3$. And have the equation:
	\begin{align*} 
	a_0 + - 3 (-b) &= 0 \\
	a_0 + a_1 + a_2 + a_3 - 7(1-b) &= 0 \\
	a_0 + 2a_1 + 4a_2 + 8a_3 - 0(2-b) &= 0 \\
	a_0 + 3a_1 + 9a_2 + 27a_3 - 2(3-b) &= 0 \\
	a_0 + 4a_1 + 16a_2 + 64a_3 - 10(4-b) &= 0 \\
	\end{align*}

	\part Solving the equation, we have $(a_0, a_1, a_2, a_3, b) = (8, 5, 6, 3, 1)$; So $Q(x) = 8 + 5x + 6 x^2 + 3 x^3$, $E(x) = x - 1$. So the first message is wrong.
	
	\part $P(x) = Q(x) / E(x) = 3x^2 + 9x + 3$. So $P(1) = 15 \equiv 4 \pmod {11}. P(1) = E$. Original message DEACK.
\end{parts}


\section{Error-Detecting Codes}
Since it's detecting not correcting, the number of symbols needed is $n + 1$, aka one more symbol to tell if there is an error occurring. There are two cases:
\begin{enumerate}[(i)]

	\item No error. Then the polynomial interpolated by the first $n$ symbols is consistent on the $(n + 1)$th point.

	\item There are errors. Then the $(n + 1)$th point is inconsistent on the polynomial no matter how many errors occurs.

\end{enumerate}

Now we show that any number lesser n + 1 is not gonna work. Say, we send n symbols, then we don't know whether errors had occurred since there is only one polynomial. Any symbol lesser n is impossible since there are n symbol needed to send.\\

{\red  
Solution:
Here I just consider when it loses only one symbol, thus making answer be n + 1. But when errors is greater than 1, it can construct a counter example, which leads the (n+1)th point is also consistent on the polynomial interpolated by the first n symbols.

Thus the correct answer is $n+k$. Still consider the polynomial interpolated by the first $n$ symbols, aka$(x_1, P(x_1)), \dots, (x_n,P(x_n))$ and also sends extra k points, from $n+1$ to $n+k$. Now we prove by cases:
\begin{enumerate}[(i)]

	\item If there are no errors, then the points from  $n+1$ to $n+k$ are all consistent by the h polynomial. h polynomials are interpolated by the received first n points.

	\item If there is an error, we prove that there are some points in the range $(n+1, n+k)$ inconsistent. If these K error are all in the $(n+1, n+k)$, then it's obviously true. In other case, if some errors occur in the range $(1, n)$, meaning there are some true uncorrupted in the range $(n+1, n+k)$, thus h polynomial won't agree on that true point. 

\end{enumerate}
And for any number lesser than n + k, the reason is same as the mistake I made. There is still possibility that all points are consistent using the error points.


}



\end{document}