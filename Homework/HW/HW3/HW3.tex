\documentclass[12pt]{exam}
\usepackage[left=3cm, right=3cm, top=3cm]{geometry}
\usepackage{amssymb}
\usepackage{amsmath}
\usepackage{mathtools}
\usepackage{enumerate}
\usepackage{xcolor}

\newcommand{\red}[0]{\color{red}}
 
\setlength{\parindent}{0pt}
\title{CS70 HW3}
\date{February 22, 2021}

\begin{document}
\maketitle
\section{Short Answer: Graphs}

\begin{parts}

	\part Connected components: 4.
	
	Just needs to draw a simple tree and because of the property: removing an edge will disconnect the tree, thus add one more connected component.
	
	{\red Connected components: 3.
	
	Removing a node means that node is not in the consideration}

	\part Number of Edges to remove: 7.
	
	Add 10 edges $\Leftrightarrow$ create 10 circles.\\
	Remove 5 edges and 3 connected components $\Leftrightarrow$ 5 = 2(disconnect) + 3(lessen a circle).\\
	So the remaining circle is 7.
	
	\part False.
	
	For $n$-dimensional hypercube: $|E| = \frac{n*(n-1)}{2}$. \\
	For a complete graph on n vertices: $|E| = \frac{n*(n-1)}{2}$.\\
	Thus edges number is equal.
	
	{\red For a complete graph on n vertices: $|E| = n*2^{n-1}$.\\
	So when n gets large, edges in $K_n$ are less than n-dimensional hypercube}.
	
	\part $\frac{n-1}{2}$
	
	One Hamiltonian cycles $\Leftrightarrow$ removes n edges. \\
	Total edges: $|E| = \frac{n*(n-1)}{2}$.\\
	Thus X Hamiltonian cycles: $|X| = \frac{n*(n-1)}{2} / n = \frac{n-1}{2}$
	
	\part Two sets: 
	\begin{enumerate}[(1).]
	
		\item \{(0,1),(1,2),(2,3),(3,4),(4,0)\}
	
		\item \{(0,3),(3,1),(1,4),(4,2),(2,0)\}
	
	\end{enumerate}


\end{parts}

\section{Eulerian Tour and Eulerian Walk}

\begin{parts}

	\part Eulerian Tour: No.
	
	From the point 3, its degree is 3, which is an odd number. And we learn that a Eulerian Tour exists if and only if the graph is even degree.

	\part Eulerian Walk: Yes.
	
	Condition: All the vertices have the even degree except two vertices. \\
	{\red Here, I only prove in the direction of $\Leftarrow$.}
	\\
	Say in $V_1, V_2, \ldots, V_n$,$V_1, V_2$ are odd degree, $V_3,  \ldots, V_n$ are even degree.\\
	Add an edge to the $V_1, V_2$, making them all even, denoted as G'. And since G' is even degree, so there is a Eulerian Tour starting at $V_1$, ending at $V_2$, and travel through the imagined edge $V_1 V_2$ back to $V_1$.\\
	Now remove the imagined edge, so there is a Eulerian Walk from $V_1$ to $V_2$.
	
	{\color{red}Now prove the direction of $\Rightarrow$\\
	Say a graph G has a Eulerian Walk from $V_1$ to $V_n$, since these two vertices only get either entered or leaved, so $V_1$ and $V_n$ are odd degree.\\
	Another condition is that when the graph has Eulerian Tour, which means all vertices have even degree.}
	
	

\end{parts}

\section{Bipartite Graph}

\begin{parts}

	\part One direction: bipartite $\implies$ no tours of odd length.
	
	Say L represents one side of vertices, $\{L_1, L_2, \ldots, L_n\}$; \\
	Say R represents one side of vertices, $\{R_1, R_2, \ldots, R_n\}$; \\
	Property: Any tour starting at L(R) side must take even length to get back to L(R) side.\\
	Proof: because there is no edge between L and L side.

	\part Another direction: no tours of odd length $\implies$ bipartite.
	
	Proof by contradiction. \\
	Say if $L_1, L_2$ is connected, any path from $L_1$ to R side and any path from R side to $L_2$ are both odd.\\
	Then there exists an odd length tour $L_1 \xRightarrow{odd} R \xRightarrow{odd} L_2 \xRightarrow{1 step} L_1$.\\
	Which contradicts to no tours of odd length.

\end{parts}

\section{Hypercubes}

\begin{parts}

	\part a little bit hard to draw with LaTex. Just use words to describe instead: \begin{enumerate}[(1).]
	
		\item 1-dimension: two dots and a line connecting them.
	
		\item 2-dimension: A square.
		
		\item 3-dimension: A cube.
	
	\end{enumerate}

	\part Proof by Induction on the number of the dimension N.
	
	\emph{Base Case}: when N = 1, two sets S and T are {0} and {1}, which consist of bipartite.
	
	\emph{Induction Hypothese}: Assume when N <= k, hypercube is bipartite. 
	
	\emph{Induction Step}: Now we wanna prove when N = k + 1, hypercube G is bipartite. We can divide the G into two parts, $G_1$ which the first bit is 0 and $G_2$ which the first bit is 1. Use induction hypothesis on $G_1$, namely sets $S_1$ and $T_1$ in $G_1$ are bipartite. By the recursive definition of hypercube, there are sets $S_2$ and $T_2$ in $G_2$ that corresponding to $S_1$ and $T_1$ in $G_1$ which are also bipartite. But since only $S_1$ and $S_2$ are connected, Set $S_1$ is not connected with $T_2$. So between two separate bipartite there are no edges, so if we def such new Set $S = S_1 \cup T_2$ and $T = T_1 \cup S_2$, set S and T are bipartite in the new N = k + 1 hypercube.
	
	so proof is done.

\end{parts}

\section{Triangulated Planar Graph}

\begin{parts}

	\part From Euler's Equation, we have equation and the fact that planar graph is triangulated: 
	\begin{align} 
	|V| + |f| &= |E| + 2  \label{eq:1} \\
	3|f| &= 2|E| \label{eq:2}
	\end{align}
	Thus from these two equation \eqref{eq:1}\eqref{eq:2}, we obtain:
	\begin{equation}
	2|E| = 6|V| - 12 \label{eq:3}
	\end{equation}
	And the total value of v is 
	\begin{align*}
	6|V| - \sum degree(v) &= 6|V| - 2|E| 
	\\
	&= 6|V| - (6|V| - 12) \\
	&= 12
	\end{align*}
	
	\part charge on degree 5 vertex : 6 - 5 = 1; \\
	charge on degree 6 vertex : 6 - 6 = 0;
	
	\part From this part, we learn we can have two cases: all vertices of degree 5 are 0 after discharge or in the reverse.
	
	\part Case 1: if all vertices of degree 5 are 0, since $\sum charges(v) = 12$ is positive, after discharge, $V_{degree = 6} = 0$, $V_{degree = 5} = 0$, $V_{degree = 7}$ has two cases: 
	\begin{enumerate}[(1).]
	
		\item $V_{degree = 7} > 0$. In this case, see part(e).
	
		\item $V_{degree = 7} \leq 0$. In this case, since all $V_{degree >= 5} \leq 0$, there must be a vertex of degree less than 5, namely 1,2,3,4, which is the first part of the claim.
	
	\end{enumerate}
	
	\part Since $V_{degree = 7} > 0$, there must be more than 5 of vertices of degree 5, namely 6 or 7 in the neighbours of that vertex. And because the graph is triangulated, so there must be 2 of these vertices of degree 5 are connected. This is exactly the second part of the claim.
	
	\part Case 2: if some vertices of degree 5 are positive, then apart from the possibility that its neighbours are all degree of 5, it's also possible that its neighbours have vertices of degree 6. But the situation in which its 5 edges adjacent to it are all vertices of degree greater than or equal to 7 are impossible. So this is the third part of the claim. 
	
\end{parts}
	
	These are all the possibilities. So the claim is proved. 

\end{document}