\documentclass[12pt]{exam}
\usepackage[left=3cm, right=3cm, top=3cm]{geometry}
\usepackage{amsmath,amsthm,amssymb}
\usepackage{enumerate}
\usepackage{xcolor}
\usepackage{mathtools}

\newcommand{\red}{\color{red}}

\setlength{\parindent}{0pt}
\title{CS70 Disc 05b}
\date{Feb 21, 2021}

\begin{document}
\maketitle
\section{Interpol Warning}

\begin{enumerate}[(1)]

	\item Linear Equations.
	
	Let $P(x) = a_3 x^3 + a_2 x^2 + a_1 x^1 + a_0$, and take four points into the equation, we have \\
	$$\begin{bmatrix}
       -1 & 1 & -1 & 1 \\
       0  &0  &0  & 1 \\
       1 & 1 & 1 & 1 \\
       8 & 4 & 2 & 1 
\end{bmatrix}
\begin{bmatrix}
       a_3  \\ a_2   \\ a_1  \\ a_0 
\end{bmatrix}
=
\begin{bmatrix}
       1  \\ 2   \\ 5  \\ 40 
\end{bmatrix}$$
Thus, solving the linear equation, we obtain 
$\begin{bmatrix}
       a_3  & a_2 &  a_1 &  a_0 
\end{bmatrix}^T$
= 
$\begin{bmatrix}
       5  & 1 &  -3 &  2 
\end{bmatrix}^T$
So the $P(x) = 5 x^3 + 1 x^2 + -3 x^1 + 2$.

	\item Lagrange Interpolation.
	\begin{align*} 
	\Delta_{-1} &= \frac{(x-0)(x-1)(x-2)}{(-1-0)(-1-1)(-1-2)} = -\frac{(x)(x-1)(x-2)}{6} \\
	\Delta_{0} &= \frac{(x+1)(x-1)(x-2)}{(0+1)(0-1)(0-2)} = \frac{(x+1)(x-1)(x-2)}{2}\\
	\Delta_{1} &= \frac{(x+1)(x)(x-2)}{(1+1)(1)(1-2)} = -\frac{(x+1)(x-1)(x-2)}{2}\\
	\Delta_{2} &= \frac{(x+1)(x)(x-1)}{(2+1)(2)(2-1)} = \frac{(x+1)(x)(x-1)}{6}
	\end{align*}
	
	The final P(x) is equal to:
	\begin{align*} 
	1 * \Delta_{-1} + 2 * \Delta_{0} + 5 * \Delta_{1} + 40 * \Delta_{2} = 5 x^3 + 1 x^2 + -3 x^1 + 2
	\end{align*} 
	which is same as method one.
	
\end{enumerate}

\section{Secrets in the United Nations}

\begin{parts}

	\part Here are two cases: 
	\begin{enumerate}[(1)]
	
		\item n countries $\implies$ n points could solve the coefficient of n-1 degree polynomial. 
	
		\item m counties + Secretary General $\implies$ give Secretary General n - m points, then the situation is same as case (1).
	
	\end{enumerate}
	So design a polynomial of degree n and produce n + (n - m) number of points, first part of n points attributes to each country and second part of (n - m) points attributes Secretary General.

	\part 
	Devine one polynomial degree of n for each country and Secretary General, and n polynomial degree of k-1 for all countries.
	\begin{enumerate}[(1)]
	
		\item For each country and Secretary General, the scheme shall be the same. And denotes that polynomial as P(x). Say each country shall have one point value, first corresponding to P(1), sec to P(2), etc.
	
		\item But for each country to obtain its attributed point value, design a polynomial degree of k-1 and generate k points for each representative, denoted as Q(x). We could set Q(0) = P(i), i is the order of that country.
	
	\end{enumerate}

\end{parts}

\section{Erasure Warm-Up}
\begin{enumerate}[(1)]

	\item Minimum q : 7 

	\item Maximum degree of polynomial : 3

\end{enumerate}


\end{document}